%!TEX root = ../csuthesis_main.tex
% 设置中文摘要
\keywordscn{中南大学;学位论文;LaTeX模板}
\categorycn{TP391}
\itemcountcn{图 \totalfigures\ 幅,表 \totaltables\ 个,参考文献 \total{citnum}\ 篇}
% \addcontentsline{toc}{section}{摘要}
\begin{abstractcn}\setlength{\baselineskip}{20pt}%\renewcommand{\baselinestretch}{1.0}
LaTeX利用设置好的模板,可以编译为格式统一的pdf。目前国内大多出版社与高校仍在使用word,word由于其强大的功能与灵活性,在新手面对形式固定的论文时,排版、编号、参考文献等简单事务反而会带来很多困难与麻烦,对于一些需要通篇修改的问题,要想达到LaTeX的效率,对word使用者来说需要具有较高的技能水平。

为了能把主要精力放在论文撰写上,许多国际期刊和高校都支持LaTeX的撰写与提交,新手不需要关心格式问题,只需要按部就班的使用少数符号标签,即可得到符合要求的文档。且在需要全篇格式修改时,更换或修改模板文件,即可直接重新编译为新的样式文档,这对于word新手使用word的感受来说是不可思议的。

本项目的目的是为了创建一个符合中南大学研究生学位论文(博士)撰写规范的TeX模板,解决学位论文撰写时格式调整的痛点。
\end{abstractcn}