%!TEX root = ../csuthesis_main.tex
% 论文正文是主体,主体部分应从另页右页开始,每一章应另起页。一般由序号标题、文字叙述、图、表格和公式等五个部分构成。
\section{绪论}

Word不难用,但是想用得漂亮还得费一番功夫。本文提供标准的学位论文 Latex 模板,让排版从此无忧。

\subsection{研究背景与意义}

\subsubsection{研究背景}

Word不难用,但是想用得漂亮还得费一番功夫。

对于小白来说,(注意!!是对于小白来说,不要跟我杠!!!我就是word小白,高端玩法玩不动):

插入个图片,下面的说明文字是不是插入文本框?那文本框要不要跟图片“组合”?,是不是直接圈没法圈起来?因为图片要变成浮动格式,和文本框绑定后再改回嵌入格式,你说蛋疼不蛋疼?不组合?那有一定概率发生你的图片在上一页,描述文字在下一页。呵呵。

插入参考文献,手动编辑?我的天哪,一百多个文献,中间插一个,怎么改序号?
很好,可以用交叉引用,一个个编辑文献格式?
很好,可以用endnote或者noteexpress的插入功能,你有没有发现插入是个宏,ctrl+z的时候会烦死你啊……
叮叮!让我们祭出LaTeX!!,有bibliography,一个 $\\cite$ 包打天下!不要太爽。

插入公式,对word小白来说,公式居中编号靠右就是一道百度搜索能力过滤器。

word里编辑三线表,啊烦躁。

等等等等……

\paragraph{哪些便利}

让我们,专心写论文好不好?

\subparagraph{参考文献}

爱你们。

等等等等……

\subparagraph{三线表}

等等等等……


\subsection{主要研究工作}
我堂堂双一流高校竟然没有官方研究生论文LaTeX模板!!!虽然我LaTeX水平也很水……但是通过大量debug也勉强给大家凑出来一个格式绝对标准的LaTeX模板,模板代码丑就丑吧,能用就行。写了大量注释,有一点LaTeX基础就可以根据自己需要修改CSUthesis.cls文件。

(1) 提供图片插入示例。

(2) 提供表格插入示例。

(3) 提供公式插入示例。

(4) 提供参考文献插入示例。

\subsection{论文组织结构}

全文内容共六章,具体内容组织如下:

第一章为绪论。

第二章为图片插入示例。

第三章为表格插入示例。

第四章为公式插入示例。

第五章为参考文献插入示例。

第六章总结与展望,总结了本文的主要工作,展望了下一阶段的研究方向。

\newpage

\section{图像布局}
\label{sec.figure}
\textbf{按学校格式要求,每个子图的小标(a)、(b)、(c)等在【左下角】。}

\subsection{单图布局}

\lipsum

\textbf{单图布局如图\ref{F.csu_single}所示。}

\begin{figure}[hbt]
\centering
\includegraphics[width=0.5\textwidth]{csu.png}
\caption{单图布局示例}
\label{F.csu_single}
\end{figure}

\subsection{横排布局}

\textbf{横排布局如图\ref{F.csu_row}所示。}子图引用如图\ref{f:subfig}。

\begin{figure}[!htb]
    \centering
    \begin{subfigure}[t]{0.24\linewidth}
        \begin{minipage}[b]{1\linewidth}
        \includegraphics[width=1\linewidth]{csu.png}
        \end{minipage}
        \caption{}
        \label{f:subfig}
    \end{subfigure}
    \begin{subfigure}[t]{0.24\linewidth}
        \begin{minipage}[b]{1\linewidth}
        \includegraphics[width=1\linewidth]{csu.png}
        \end{minipage}
        \caption{}
    \end{subfigure}
    \begin{subfigure}[t]{0.24\linewidth}
        \begin{minipage}[b]{1\linewidth}
        \includegraphics[width=1\linewidth]{csu.png}
        \end{minipage}
        \caption{}
    \end{subfigure}
    \begin{subfigure}[t]{0.24\linewidth}
        \begin{minipage}[b]{1\linewidth}
        \includegraphics[width=1\linewidth]{csu.png}
        \end{minipage}
        \caption{}
    \end{subfigure}
    \caption{横排布局示例}
    \label{F.csu_row}
\end{figure}

\lipsum

\subsection{竖排布局}
\textbf{竖排布局如图\ref{F.csu_col}所示。}

\begin{figure}[!htb]
    \centering
    \begin{subfigure}[t]{0.15\linewidth}
        \begin{minipage}[b]{1\linewidth}
        \includegraphics[width=1\linewidth]{csu.png}
        \end{minipage}
        \caption{}
    \end{subfigure}\\
    \begin{subfigure}[t]{0.15\linewidth}
        \begin{minipage}[b]{1\linewidth}
        \includegraphics[width=1\linewidth]{csu.png}
        \end{minipage}
        \caption{}
    \end{subfigure}
    \caption{竖排布局示例}
    \label{F.csu_col}
\end{figure}

\lipsum

\subsection{竖排多图横排布局}

\begin{figure}[!htb]
    \centering
    \begin{subfigure}[t]{0.13\linewidth}
        \begin{minipage}[b]{1\linewidth}
        \includegraphics[width=1\linewidth]{csu.png} \vspace{-1ex} \vfill
        \includegraphics[width=1\linewidth]{csu.png}
        \end{minipage}
        \caption{}
    \end{subfigure}
    \begin{subfigure}[t]{0.13\linewidth}
        \begin{minipage}[b]{1\linewidth}
        \includegraphics[width=1\linewidth]{csu.png} \vspace{-1ex} \vfill
        \includegraphics[width=1\linewidth]{csu.png}
        \end{minipage}
        \caption{}
    \end{subfigure}
    \caption{竖排多图横排布局}
    \label{F.csu_col_row}
\end{figure}

\textbf{竖排多图横排布局如图\ref{F.csu_col_row}所示。注意看(a)、(b)编号与图关系。}


\subsection{横排多图竖排布局}

\lipsum

\begin{figure}[!htb]
    \centering
    \begin{subfigure}[t]{0.3\linewidth}
        \begin{minipage}[b]{1\linewidth}
        \includegraphics[width=0.45\linewidth]{csu.png}
        \includegraphics[width=0.45\linewidth]{csu.png}
        \end{minipage}
        \caption{}
    \end{subfigure}\\
    \begin{subfigure}[t]{0.3\linewidth}
        \begin{minipage}[b]{1\linewidth}
        \includegraphics[width=0.45\linewidth]{csu.png}
        \includegraphics[width=0.45\linewidth]{csu.png}
        \end{minipage}
        \caption{}
    \end{subfigure}
    \caption{横排多图竖排布局}
    \label{F.csu_row_col}
\end{figure}

\textbf{横排多图竖排布局如图\ref{F.csu_row_col}所示。注意看(a)、(b)编号与图关系。}

\subsection{本章小结}
本章示例图片布局。
\subsubsection{三级标题}

\newpage


\section{表格插入示例}

\begin{table}[htb]
  \centering
  \caption{学校文件里对表格的要求不是很高,不过按照学术论文的一般规范,表格为三线表。}
  \label{T.example}
  \begin{tabular}{llllll}
  \hline
   & A  & B  & C  & D  & E \\
  \hline
1 	& 212 & 414 & 4 		& 23 & fgw	\\
2 	& 212 & 414 & v 		& 23 & fgw	\\
3 	& 212 & 414 & vfwe		& 23 & 嗯	\\
4 	& 212 & 414 & 4fwe		& 23 & 嗯	\\
5 	& af2 & 4vx & 4 		& 23 & fgw	\\
6 	& af2 & 4vx & 4 		& 23 & fgw	\\
7 	& 212 & 414 & 4 		& 23 & fgw	\\

\hline{}
\end{tabular}
\end{table}

\textbf{表格如表\ref{T.example}所示,latex表格技巧很多,这里不再详细介绍。}

OK,再举个例子,表格并列。若太宽,还可以用`adjustbox`加以旋转,如表\ref{table:1b}。

\begin{table}[htp]
	\centering
	\begin{subtable}[t]{2in}
		\centering
		\caption{标题1}\label{table:1a}
		\begin{tabular}{|l|l|l|}
		\hline
		100 & 200 & 300\\
		\hline
		400 & 500 & 600\\
		\hline
		\end{tabular}
	\end{subtable}
	\quad
	\begin{subtable}[t]{2in}
		\centering
		\caption{标题2}\label{table:1b}
		\begin{adjustbox}{angle=90}
		\begin{tabular}{|l|l|l|}
		\hline
		100 & 200 & 300\\
		\hline
		400 & 500 & 600\\
		\hline
		\end{tabular}
		\end{adjustbox}
	\end{subtable}
	\caption{主表标题}\label{table:1}
\end{table}

\lipsum

\newpage

\section{公式插入示例}

\lipsum

\textbf{公式插入示例如公式(\ref{E.example})所示。}

\begin{equation}
\gamma_{x}=
\left\{
  \begin{array}{lr}
  0, & {\rm if}~~\;|x| \leq \delta \\
  x, & {\rm otherwise}
  \end{array}
\right.
\label{E.example}
\end{equation}


\newpage

\section{算法插入示例}

如算法\ref{alg:myalgorithm}第\ref{code:someline}行所示,……
\begin{algorithm}
\caption{algorithm caption} %算法的名字
\hspace*{0.02in} \textbf{Input:} Input parameters A, B, C.\\
\hspace*{0.02in} \textbf{Initialization:} Initialization parameters.\\
\hspace*{0.02in} \textbf{Output:} Output result.
\begin{algorithmic}[1]
	\State some description % \State 后写一般语句
	\State \textbf{If} {condition} \textbf{then}
	\label{code:someline}  
	 	\State \quad {coding;}
	\State \textbf{Else}
	 	\State \quad {coding;}
	 	\State \quad {coding;}
	\State \textbf{EndIf}
	\State \textbf{Return} {result}
\end{algorithmic}
\label{alg:myalgorithm}
\end{algorithm}

\newpage

\section{参考文献插入示例}

LaTeX\cite{lamport1994latex}插入参考文献最方便的方式是使用bibliography\cite{pritchard1969statistical},大多数出版商的论文页面都会有导出bib格式参考文献的链接,把每个文献的bib放入``thesis-references'',然后用bibkey即可插入参考文献。

\lipsum

\newpage


\section{总结与展望}

\noindent{纯数字编号}
\begin{enumerate}[itemindent=2em]
 \item XXXXXXXXXX
 \label{item1}
 \item XXXXXXXXXX
 \item XXXXXXXXXX
\end{enumerate}
罗马编号
\begin{enumerate}[label=(\roman*),itemindent=2em]
 \item XXXXXXXXXX
 \label{item2}
 \item XXXXXXXXXX
 \item XXXXXXXXXX
\end{enumerate}
括号编号
\begin{enumerate}[label=(\arabic*),itemindent=2em]
 \item XXXXXXXXXX
 \label{item3}
 \item XXXXXXXXXX
 \item XXXXXXXXXX
\end{enumerate}
半括号编号
\begin{enumerate}[label=\arabic*),itemindent=2em]
 \item XXXXXXXXXX
 \label{item4}
 \item XXXXXXXXXX
 \item XXXXXXXXXX
\end{enumerate}
小字母编号
\begin{enumerate}[label=\alph*),itemindent=2em]
 \item XXXXXXXXXX
 \label{item5}
 \item XXXXXXXXXX
 \item XXXXXXXXXX
\end{enumerate}

引用测试,正如\ref{item1}、\ref{item2}、\ref{item3}、\ref{item4}、\ref{item5}所示

\subsection{工作展望}
手动编号 %(不推荐,无法被交叉引用)
\par
本课题针对XX,鉴于XXX,对XX进行了提高,但是XXX,所以有如下XX:

(1)目前XX虽然XX,但是XX仍然XX,所以XX仍然是一个值得XX的问题。

(2)随着XX,XX具有XX的问题,仍值得进一步XX。

(3)本课题在XX有了XX,但是XX的XX还存在XX,所以XX。


\newpage
