\documentclass{CSUthesis}

\usepackage{algorithm}
\usepackage{algorithmicx}

%算法
%\begin{algorithm}[t]
%\caption{algorithm caption} %算法的名字
%\hspace*{0.02in} {\bf Input:} %算法的输入, \hspace*{0.02in}用来控制位置,同时利用 \\ 进行换行
%input parameters A, B, C\\
%\hspace*{0.02in} {\bf Output:} %算法的结果输出
%output result
%\begin{algorithmic}[1]
%\State some description % \State 后写一般语句
%\For{condition} % For 语句,需要和EndFor对应
%	\State ...
%	\If{condition} % If 语句,需要和EndIf对应
%		\State ...
%	\Else
%		\State ...
%	\EndIf
%\EndFor
%\While{condition} % While语句,需要和EndWhile对应
%	\State ...
%\EndWhile
%\State \Return result
%\end{algorithmic}
%\end{algorithm}

%!TEX root = ../csuthesis_main.tex
% 文章信息
\titlecn{中南大学研究生学位论文LaTeX模板}
\titleen{LaTeX Template of Postgraduate Thesis of \\Central South University}

\priormajor{计算机科学与技术}
\minormajor{计算机应用技术}
\interestmajor{旁门左道}
\author{某大侠}
\supervisor{某老侠\ 教授}
\subsupervisor{}
\department{计算机学院}
\studentid{144601044}
\thesisdate{year=2019,month=5}



\clcnumber{TP391} 				% 中图分类号 Chinese Library Classification
\schoolcode{10533}			% 学校代码
\udc{004.9}						% UDC
\academiccategory{学术学位}	% 学术类别


\newif \ifblindreview % 条件语句,是否是盲审版本
% \blindreviewtrue
\blindreviewfalse


% lipsum
\newcommand{\lipsum} {
    
这是一段随机插入的文本,用来填充模板布局,感受模板视觉效果。

中南大学坐落在中国历史文化名城──湖南省长沙市,占地面积5886亩,建筑面积276万平方米,跨湘江两岸,依巍巍岳麓,临滔滔湘水,环境幽雅,景色宜人,是求知治学的理想园地。
    
中南大学是教育部直属全国重点大学、国家“211工程”首批重点建设高校、国家“985工程”部省重点共建高水平大学和国家“2011计划”首批牵头高校,2017年9月经国务院批准入选世界一流大学A类建设高校。现任校党委书记易红、校长田红旗。

中南大学由原湖南医科大学、长沙铁道学院与中南工业大学于2000年4月合并组建而成。原中南工业大学的前身为创建于1952年的中南矿冶学院,原长沙铁道学院的前身为创建于1953年的中南土木建筑学院,两校的主体学科最早溯源于1903年创办的湖南高等实业学堂的矿科和路科。原湖南医科大学的前身为1914年创建的湘雅医学专门学校,是我国创办最早的西医高等学校之一。

中南大学秉承百年办学积淀,顺应中国高等教育体制改革大势,弘扬以“知行合一、经世致用”为核心的大学精神,力行“向善、求真、唯美、有容”的校风,坚持自身办学特色,服务国家和社会重大需求,团结奋进,改革创新,追求卓越,综合实力和整体水平大幅提升。

这是一段随机插入的文本,用来填充模板布局,感受模板视觉效果。
}

\begin{document}
%%%%%%%%%%%%%%%%%%%%%%%%%%%%%%%%%%%%%%%%%%%%%%%%%%
% 封面
% -----------------------------------------------%
\makecoverpage

%%%%%%%%%%%%%%%%%%%%%%%%%%%%%%%%%%%%%%%%%%%%%%%%%%
% 前置部分的页眉页脚设置
% -----------------------------------------------%
\newpage
% 正文和后置部分用阿拉伯数字编连续码,前置部分用罗马数字单独编连续码(封面除外)。
% 设置封面页后的页码
\pagenumbering{Roman} % 大写罗马字母
\setcounter{page}{1} % 从1开始编号页码
% 设置页眉和页脚 
\pagestyle{fancy}
% 正文以前部分无需页眉
\fancyhf{} % 清空原有格式
\renewcommand{\headrulewidth}{0pt}
% 封面页无需页码,其他前置部分需要(按此理解扉页也是要页码的)。
\fancyhf[CF]{\thepage} % 所有(奇数和偶数)中间页脚

\ifblindreview	% 盲审不需要扉页和声明页
\else
%%%%%%%%%%%%%%%%%%%%%%%%%%%%%%%%%%%%%%%%%%%%%%%%%%
% 扉页 
% -----------------------------------------------%
\maketitlepage
\newpage

%%%%%%%%%%%%%%%%%%%%%%%%%%%%%%%%%%%%%%%%%%%%%%%%%%
% 声明页
% -----------------------------------------------%
\announcement
\newpage
\fi
%%%%%%%%%%%%%%%%%%%%%%%%%%%%%%%%%%%%%%%%%%%%%%%%%%
% 中文摘要
% -----------------------------------------------%
%!TEX root = ../csuthesis_main.tex
% 设置中文摘要
\keywordscn{中南大学;学位论文;LaTeX模板}
\categorycn{TP391}
\addcontentsline{toc}{section}{摘要}
\begin{abstractcn}
LaTeX利用设置好的模板,可以编译为格式统一的pdf。目前国内大多出版社与高校仍在使用word,word由于其强大的功能与灵活性,在新手面对形式固定的论文时,排版、编号、参考文献等简单事务反而会带来很多困难与麻烦,对于一些需要通篇修改的问题,要想达到LaTeX的效率,对word使用者来说需要具有较高的技能水平。

为了能把主要精力放在论文撰写上,许多国际期刊和高校都支持LaTeX的撰写与提交,新手不需要关心格式问题,只需要按部就班的使用少数符号标签,即可得到符合要求的文档。且在需要全篇格式修改时,更换或修改模板文件,即可直接重新编译为新的样式文档,这对于word新手使用word的感受来说是不可思议的。

本项目的目的是为了创建一个符合中南大学研究生学位论文(博士)撰写规范的TeX模板,解决学位论文撰写时格式调整的痛点。


\noindent 图X幅,表X个,参考文献X篇(四号宋体)

\end{abstractcn}
\newpage

%%%%%%%%%%%%%%%%%%%%%%%%%%%%%%%%%%%%%%%%%%%%%%%%%%
% 英文摘要
% -----------------------------------------------%
%!TEX root = ../csuthesis_main.tex
\keywordsen{CSU;~LaTeX;~Template}
\categoryen{TP391}
\itemcountcn{There are \totalfigures\ figures, \totaltables\ tables, and \total{citnum}\ citations in this thesis.}
% \addcontentsline{toc}{section}{ABSTRACT}
\begin{abstracten}\setlength{\baselineskip}{20pt}
LaTeX can be compiled into a pdf of uniform format using the set template.At present, most domestic publishers and universities still use word. Because of its powerful function and flexibility, when faced with fixed-form papers by novices, simple matters such as typesetting, numbering, and reference documents will bring many difficulties and troubles. For some problems that need to be modified throughout, to achieve the efficiency of LaTeX, it requires a high level of skill for word users.

In order to focus on the writing of papers, many international journals and universities support the writing and submission of LaTeX. Novices don't need to care about formatting issues. They only need to use a few symbolic labels step by step to get the documents that meet the requirements. And when you need to modify the entire format, you can directly recompile the template file by replacing or modifying the template file. This is incredible for the word novice to use the word.

The purpose of this project is to create a TeX template that meets the specifications of the graduate degree thesis (PhD) of Central South University, and to address the pain points of format adjustment during the dissertation writing.
\end{abstracten}
\newpage

%%%%%%%%%%%%%%%%%%%%%%%%%%%%%%%%%%%%%%%%%%%%%%%%%%
% 正文页眉页脚
% -----------------------------------------------%
\setcounter{page}{1} % 重置目录页码为小写罗马字体
\pagenumbering{roman} % 设置页码为小写罗马字体
% 设置页眉和页脚 %
\pagestyle{fancy}
\fancyhf[CF]{\thepage} % 所有(奇数和偶数)中间页脚

% 目录
% -------------------------------------------%
{
\renewcommand{\contentsname}{\hfill \heiti \zihao{3} 目\quad 录\hfill}  
	\renewcommand*{\baselinestretch}{1.2}   % 行间距
    \tableofcontents
}
\newpage

\renewcommand{\headrulewidth}{1pt}

% 去掉页眉章节序号后面的“.”
\renewcommand{\sectionmark}[1]{\markboth{第{\thesection}章~ #1}{第{\thesection}章~ #1}}
\renewcommand{\subsectionmark}[1]{\markright{\leftmark}}
\renewcommand{\subsubsectionmark}[1]{\markright{\leftmark}}

\fancyhf[RH]{\songti \zihao{5} \leftmark} % 设置所有(奇数和偶数)右侧页眉net
\fancyhf[LH]{\songti \zihao{5} 中南大学博士学位论文} % 设置所有(奇数和偶数)左侧页眉
% 正文内容 
% --------------------------------------------%
\setcounter{page}{1} % 重置页码编号
\pagenumbering{arabic} % 设置页码编号为阿拉伯数字

% 可以使用include命令导入tex文件,从而避免过多修改本文件。

% 论文正文是主体,主体部分应从另页右页开始,每一章应另起页。一般由序号标题、文字叙述、图、表格和公式等五个部分构成。

% 重新设置正文行间距,因为前置部分设置时候行间距被改过
\renewcommand*{\baselinestretch}{1.0}   % 几倍行间距
\setlength{\baselineskip}{20pt}         % 基准行间距

% 正文
{
% 表格字号应比正文小,一般五号/10.5pt,但是暂时没法再cls里设置(不然会影响到封面等tabular环境)
% 所以目前只好在主文件里局部\AtBeginEnvironment
	\AtBeginEnvironment{tabular}{\zihao{5}}
	%!TEX root = ../csuthesis_main.tex
% 论文正文是主体,主体部分应从另页右页开始,每一章应另起页。一般由序号标题、文字叙述、图、表格和公式等五个部分构成。
\section{绪论}

Word不难用,但是想用得漂亮还得费一番功夫。本文提供标准的学位论文 Latex 模板,让排版从此无忧。

\subsection{研究背景与意义}

\subsubsection{研究背景}

Word不难用,但是想用得漂亮还得费一番功夫。

对于小白来说,(注意!!是对于小白来说,不要跟我杠!!!我就是word小白,高端玩法玩不动):

插入个图片,下面的说明文字是不是插入文本框?那文本框要不要跟图片“组合”?,是不是直接圈没法圈起来?因为图片要变成浮动格式,和文本框绑定后再改回嵌入格式,你说蛋疼不蛋疼?不组合?那有一定概率发生你的图片在上一页,描述文字在下一页。呵呵。

插入参考文献,手动编辑?我的天哪,一百多个文献,中间插一个,怎么改序号?
很好,可以用交叉引用,一个个编辑文献格式?
很好,可以用endnote或者noteexpress的插入功能,你有没有发现插入是个宏,ctrl+z的时候会烦死你啊……
叮叮!让我们祭出LaTeX!!,有bibliography,一个 $\\cite$ 包打天下!不要太爽。

插入公式,对word小白来说,公式居中编号靠右就是一道百度搜索能力过滤器。

word里编辑三线表,啊烦躁。

等等等等……

\paragraph{哪些便利}

让我们,专心写论文好不好?

\subparagraph{参考文献}

爱你们。

等等等等……

\subparagraph{三线表}

等等等等……


\subsection{主要研究工作}
我堂堂双一流高校竟然没有官方研究生论文LaTeX模板!!!虽然我LaTeX水平也很水……但是通过大量debug也勉强给大家凑出来一个格式绝对标准的LaTeX模板,模板代码丑就丑吧,能用就行。写了大量注释,有一点LaTeX基础就可以根据自己需要修改CSUthesis.cls文件。

(1) 提供图片插入示例。

(2) 提供表格插入示例。

(3) 提供公式插入示例。

(4) 提供参考文献插入示例。

\subsection{论文组织结构}

全文内容共六章,具体内容组织如下:

第一章为绪论。

第二章为图片插入示例。

第三章为表格插入示例。

第四章为公式插入示例。

第五章为参考文献插入示例。

第六章总结与展望,总结了本文的主要工作,展望了下一阶段的研究方向。

\newpage

\section{图像布局}
\label{sec.figure}
\textbf{按学校格式要求,每个子图的小标(a)、(b)、(c)等在【左下角】。}

\subsection{单图布局}

\lipsum

\textbf{单图布局如图\ref{F.csu_single}所示。}

\begin{figure}[hbt]
\centering
\includegraphics[width=0.5\textwidth]{csu.png}
\caption{单图布局示例}
\label{F.csu_single}
\end{figure}

\subsection{横排布局}

\textbf{横排布局如图\ref{F.csu_row}所示。}子图引用如图\ref{f:subfig}。

\begin{figure}[!htb]
    \centering
    \begin{subfigure}[t]{0.24\linewidth}
        \begin{minipage}[b]{1\linewidth}
        \includegraphics[width=1\linewidth]{csu.png}
        \end{minipage}
        \caption{}
        \label{f:subfig}
    \end{subfigure}
    \begin{subfigure}[t]{0.24\linewidth}
        \begin{minipage}[b]{1\linewidth}
        \includegraphics[width=1\linewidth]{csu.png}
        \end{minipage}
        \caption{}
    \end{subfigure}
    \begin{subfigure}[t]{0.24\linewidth}
        \begin{minipage}[b]{1\linewidth}
        \includegraphics[width=1\linewidth]{csu.png}
        \end{minipage}
        \caption{}
    \end{subfigure}
    \begin{subfigure}[t]{0.24\linewidth}
        \begin{minipage}[b]{1\linewidth}
        \includegraphics[width=1\linewidth]{csu.png}
        \end{minipage}
        \caption{}
    \end{subfigure}
    \caption{横排布局示例}
    \label{F.csu_row}
\end{figure}

\lipsum

\subsection{竖排布局}
\textbf{竖排布局如图\ref{F.csu_col}所示。}

\begin{figure}[!htb]
    \centering
    \begin{subfigure}[t]{0.15\linewidth}
        \begin{minipage}[b]{1\linewidth}
        \includegraphics[width=1\linewidth]{csu.png}
        \end{minipage}
        \caption{}
    \end{subfigure}\\
    \begin{subfigure}[t]{0.15\linewidth}
        \begin{minipage}[b]{1\linewidth}
        \includegraphics[width=1\linewidth]{csu.png}
        \end{minipage}
        \caption{}
    \end{subfigure}
    \caption{竖排布局示例}
    \label{F.csu_col}
\end{figure}

\lipsum

\subsection{竖排多图横排布局}

\begin{figure}[!htb]
    \centering
    \begin{subfigure}[t]{0.13\linewidth}
        \begin{minipage}[b]{1\linewidth}
        \includegraphics[width=1\linewidth]{csu.png} \vspace{-1ex} \vfill
        \includegraphics[width=1\linewidth]{csu.png}
        \end{minipage}
        \caption{}
    \end{subfigure}
    \begin{subfigure}[t]{0.13\linewidth}
        \begin{minipage}[b]{1\linewidth}
        \includegraphics[width=1\linewidth]{csu.png} \vspace{-1ex} \vfill
        \includegraphics[width=1\linewidth]{csu.png}
        \end{minipage}
        \caption{}
    \end{subfigure}
    \caption{竖排多图横排布局}
    \label{F.csu_col_row}
\end{figure}

\textbf{竖排多图横排布局如图\ref{F.csu_col_row}所示。注意看(a)、(b)编号与图关系。}


\subsection{横排多图竖排布局}

\lipsum

\begin{figure}[!htb]
    \centering
    \begin{subfigure}[t]{0.3\linewidth}
        \begin{minipage}[b]{1\linewidth}
        \includegraphics[width=0.45\linewidth]{csu.png}
        \includegraphics[width=0.45\linewidth]{csu.png}
        \end{minipage}
        \caption{}
    \end{subfigure}\\
    \begin{subfigure}[t]{0.3\linewidth}
        \begin{minipage}[b]{1\linewidth}
        \includegraphics[width=0.45\linewidth]{csu.png}
        \includegraphics[width=0.45\linewidth]{csu.png}
        \end{minipage}
        \caption{}
    \end{subfigure}
    \caption{横排多图竖排布局}
    \label{F.csu_row_col}
\end{figure}

\textbf{横排多图竖排布局如图\ref{F.csu_row_col}所示。注意看(a)、(b)编号与图关系。}

\subsection{本章小结}
本章示例图片布局。
\subsubsection{三级标题}

\newpage


\section{表格插入示例}

\begin{table}[htb]
  \centering
  \caption{学校文件里对表格的要求不是很高,不过按照学术论文的一般规范,表格为三线表。}
  \label{T.example}
  \begin{tabular}{llllll}
  \hline
   & A  & B  & C  & D  & E \\
  \hline
1 	& 212 & 414 & 4 		& 23 & fgw	\\
2 	& 212 & 414 & v 		& 23 & fgw	\\
3 	& 212 & 414 & vfwe		& 23 & 嗯	\\
4 	& 212 & 414 & 4fwe		& 23 & 嗯	\\
5 	& af2 & 4vx & 4 		& 23 & fgw	\\
6 	& af2 & 4vx & 4 		& 23 & fgw	\\
7 	& 212 & 414 & 4 		& 23 & fgw	\\

\hline{}
\end{tabular}
\end{table}

\textbf{表格如表\ref{T.example}所示,latex表格技巧很多,这里不再详细介绍。}

OK,再举个例子,表格并列。若太宽,还可以用`adjustbox`加以旋转,如表\ref{table:1b}。

\begin{table}[htp]
	\centering
	\begin{subtable}[t]{2in}
		\centering
		\caption{标题1}\label{table:1a}
		\begin{tabular}{|l|l|l|}
		\hline
		100 & 200 & 300\\
		\hline
		400 & 500 & 600\\
		\hline
		\end{tabular}
	\end{subtable}
	\quad
	\begin{subtable}[t]{2in}
		\centering
		\caption{标题2}\label{table:1b}
		\begin{adjustbox}{angle=90}
		\begin{tabular}{|l|l|l|}
		\hline
		100 & 200 & 300\\
		\hline
		400 & 500 & 600\\
		\hline
		\end{tabular}
		\end{adjustbox}
	\end{subtable}
	\caption{主表标题}\label{table:1}
\end{table}

\lipsum

\newpage

\section{公式插入示例}

\lipsum

\textbf{公式插入示例如公式(\ref{E.example})所示。}

\begin{equation}
\gamma_{x}=
\left\{
  \begin{array}{lr}
  0, & {\rm if}~~\;|x| \leq \delta \\
  x, & {\rm otherwise}
  \end{array}
\right.
\label{E.example}
\end{equation}


\newpage

\section{算法插入示例}

如算法\ref{alg:myalgorithm}第\ref{code:someline}行所示,……
\begin{algorithm}
\caption{algorithm caption} %算法的名字
\hspace*{0.02in} \textbf{Input:} Input parameters A, B, C.\\
\hspace*{0.02in} \textbf{Initialization:} Initialization parameters.\\
\hspace*{0.02in} \textbf{Output:} Output result.
\begin{algorithmic}[1]
	\State some description % \State 后写一般语句
	\State \textbf{If} {condition} \textbf{then}
	\label{code:someline}  
	 	\State \quad {coding;}
	\State \textbf{Else}
	 	\State \quad {coding;}
	 	\State \quad {coding;}
	\State \textbf{EndIf}
	\State \textbf{Return} {result}
\end{algorithmic}
\label{alg:myalgorithm}
\end{algorithm}

\newpage

\section{参考文献插入示例}

LaTeX\cite{lamport1994latex}插入参考文献最方便的方式是使用bibliography\cite{pritchard1969statistical},大多数出版商的论文页面都会有导出bib格式参考文献的链接,把每个文献的bib放入``thesis-references'',然后用bibkey即可插入参考文献。

\lipsum

\newpage


\section{总结与展望}

\noindent{纯数字编号}
\begin{enumerate}[itemindent=2em]
 \item XXXXXXXXXX
 \label{item1}
 \item XXXXXXXXXX
 \item XXXXXXXXXX
\end{enumerate}
罗马编号
\begin{enumerate}[label=(\roman*),itemindent=2em]
 \item XXXXXXXXXX
 \label{item2}
 \item XXXXXXXXXX
 \item XXXXXXXXXX
\end{enumerate}
括号编号
\begin{enumerate}[label=(\arabic*),itemindent=2em]
 \item XXXXXXXXXX
 \label{item3}
 \item XXXXXXXXXX
 \item XXXXXXXXXX
\end{enumerate}
半括号编号
\begin{enumerate}[label=\arabic*),itemindent=2em]
 \item XXXXXXXXXX
 \label{item4}
 \item XXXXXXXXXX
 \item XXXXXXXXXX
\end{enumerate}
小字母编号
\begin{enumerate}[label=\alph*),itemindent=2em]
 \item XXXXXXXXXX
 \label{item5}
 \item XXXXXXXXXX
 \item XXXXXXXXXX
\end{enumerate}

引用测试,正如\ref{item1}、\ref{item2}、\ref{item3}、\ref{item4}、\ref{item5}所示

\subsection{工作展望}
手动编号 %(不推荐,无法被交叉引用)
\par
本课题针对XX,鉴于XXX,对XX进行了提高,但是XXX,所以有如下XX:

(1)目前XX虽然XX,但是XX仍然XX,所以XX仍然是一个值得XX的问题。

(2)随着XX,XX具有XX的问题,仍值得进一步XX。

(3)本课题在XX有了XX,但是XX的XX还存在XX,所以XX。


\newpage

}

%%%%%%%%%%%%%%%%%%%%%%%%%%%%%%%%%%%%%%%%%%%%%%%%%%
% 临时标签,用于编译时追踪正文末尾
%%%%%%%%%%%%%%%%%%%%%%%%%%%%%%%%%%%%%%%%%%%%%%%%%%

%%%%%%%%%%%%%%%%%%%%%%%%%%%%%%%%%%%%%%%%%%%%%%%%%%
% 后续内容,标题三号黑体居中,章节无编号
% --------------------------------------------%

% https://www.zhihu.com/question/29413517/answer/44358389 %
% 说明如下:
% secnumdepth 这个计数器是 LaTeX 标准文档类用来控制章节编号深度的。在 article 中,这个计数器的值默认是 3,对应的章节命令是 \subsubsection。也就是说,默认情况下,article 将会对 \subsubsection 及其之上的所有章节标题进行编号,也就是 \part, \section, \subsection, \subsubsection。LaTeX 标准文档类中,最大的标题是 \part。它在 book 和 report 类中的层级是「-1」,在 article 类中的层级是「0」。这里,我们在调用 \appendix 的时候将计数器设置为 -2,因此所有的章节命令都不会编号了。不过,一般还是会保留 \part 的编号的。所以在实际使用中,将它设置为 0 就可以了。

% 在修改过程中请注意不要破环命令的完整性

\renewcommand\appendix{\setcounter{secnumdepth}{-2}}
\appendix

% 主文件有代码去掉页眉章节编号的“.”,但这会因为bug导致无编号章节显示一个错误编号,所以这里在无编号章节之前再次重定义sectionmark。
\renewcommand{\sectionmark}[1]{\markboth{#1}{#1}}
\renewcommand{\subsectionmark}[1]{\markright{\leftmark}}
\renewcommand{\subsubsectionmark}[1]{\markright{\leftmark}}

% section 标题从这里往后改为三号黑体居中
\titleformat{\section}{\centering \zihao{3}\heiti}{\thesection}{1em}{}

% \section{参考文献} % bibliography会自动显示参考文献四个字
\addcontentsline{toc}{section}{参考文献} % 由于参考文献不是section,这句把参考文献加入目录
% \nocite{*} % 该命令用于显示全部参考文献,即使文中没引用
% cls文件中已经引入package,这里不需要调用 \bibliographystyle 了。
% \bibliographystyle{gbt7714-2005} 
\bibliography{thesis-references}
\newpage

%!TEX root = ../csuthesis_main.tex
{~}
\vspace{-9pt}
\section{攻读学位期间主要研究成果} % 无章节编号

\ifblindreview
% \noindent
% (盲审隐去作者相关具体信息)
\fi
\vspace{11pt}
\subsection*{一、学术论文}

\ifblindreview

% \noindent
% 第一作者:JCR 1 区 x 篇,会议 x 篇 \\{}
% 第二作者:JCR 1 区 x 篇,3 区 x 篇,4 区 x 篇,EI x 篇 

% \noindent
% 投稿状态: 
% IEEE Transactions on Image Processing 1篇(under review)\\{}
% IEEE Transactions on Circuits and Systems for Video Technology 1 篇(Accept with Minor Revision)

% 学位办老师要求用如下这种几乎算是单盲的格式,我也木有办法……
\subsubsection*{已录用/检索论文}
%\noindent
第一作者:
\begin{enumerate}[label={[\arabic*]},itemindent=2em,wide]
\item CSU Latex Template[J]. CSU player: 1(1):1-10. {\bfseries \heiti(SCI 录用,JCR 1 区)}
\item CSU Latex Template[J]. CSU player: 1(1):1-10. {\bfseries \heiti(SCI 检索,JCR 2 区)}
\end{enumerate}
第二作者:
\begin{enumerate}[label={[\arabic*]},itemindent=2em,wide]
\item CSU Latex Template[J]. CSU player: 1(1):1-10. {\bfseries \heiti(SCI 检索,JCR 1 区)}
\item CSU Latex Template[J]. CSU player: 1(1):1-10. {\bfseries \heiti(SCI 检索,JCR 2 区)}
\end{enumerate}
%\noindent
\subsubsection*{投稿状态论文}
%\noindent
第一作者:
\begin{enumerate}[label={[\arabic*]},itemindent=2em,wide]
\item CSU Latex Template. XXX Transactions on CSU player. {\bfseries \heiti(SCI Under Review,JCR 1 区)}
\end{enumerate}
第二作者:
\begin{enumerate}[label={[\arabic*]},itemindent=2em,wide]
\item CSU Latex Template. XXX Transactions on CSU player. {\bfseries \heiti(SCI Under Review,JCR 1 区)}
\item CSU Latex Template. XXX Transactions on CSU player. {\bfseries \heiti(SCI Under Review,JCR 2 区)}
\end{enumerate}


\else
% 标准版本
\begin{enumerate}[label={[\arabic*]},itemindent=2em,wide]
\item \textbf{Daxia Mou}, Director, Someone. CSU Latex Template[J]. CSU player: 1(1):1-10. {\bfseries \heiti(SCI 检索,JCR 1 区)}
\item Director, \textbf{Daxia Mou}, Someone, Someother. XXXXXX[J]. Transactions on Image Processing. {\bfseries \heiti(SCI Under Review,JCR 1 区)}
\item Director, \textbf{Daxia Mou}, Someone, Someother. XXXXXX[J]. Transactions on Circuits and Systems for Video Technology. {\bfseries \heiti(SCI Under Review,JCR 1 区)}
\end{enumerate}
\fi

\vspace{22pt}
\subsection*{二、发明专利}
\ifblindreview
发明专利 2 项,已公开
\else
\begin{enumerate}[label={[\arabic*]},itemindent=2em,wide]
\item 某大侠,XXX,XXX. 一种用Latex写中南大学学位论文的方法. 申请号:CN20190415xxxx,公开号:CNXXXXXXXXXA
\end{enumerate}
\fi

\ifblindreview
\else

\vspace{22pt}
\subsection*{三、主持和参与的科研项目}
\begin{enumerate}[label={[\arabic*]},itemindent=2em,wide]
\item 国家自然科学基金面上项目《XXXXXXXXXXXX》, 项目编号:XXXXXXXX,参与.
\end{enumerate}

% {~}
\vspace{22pt}
\subsection*{四、个人获奖情况}
%\noindent
\begin{enumerate}[label={[\arabic*]},itemindent=2em,wide]
\item XX金奖
\item XX奖学金
\end{enumerate}
\fi

\newpage

\ifblindreview
\else
\fancyhf[RH]{\songti \zihao{5} 致谢}
{~}
\vspace{-9pt}
\section{致~~~~谢} % 无章节编号
\lipsum
\newpage
\fi

\end{document}
